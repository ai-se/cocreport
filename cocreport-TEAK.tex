% This is "sig-alternate.tex" V2.0 May 2012
% This file should be compiled with V2.5 of "sig-alternate.cls" May 2012
%
% This example file demonstrates the use of the 'sig-alternate.cls'
% V2.5 LaTeX2e document class file. It is for those submitting
% articles to ACM Conference Proceedings WHO DO NOT WISH TO
% STRICTLY ADHERE TO THE SIGS (PUBS-BOARD-ENDORSED) STYLE.
% The 'sig-alternate.cls' file will produce a similar-looking,
% albeit, 'tighter' paper resulting in, invariably, fewer pages.
%
% ----------------------------------------------------------------------------------------------------------------
% This .tex file (and associated .cls V2.5) produces:
%       1) The Permission Statement
%       2) The Conference (location) Info information
%       3) The Copyright Line with ACM data
%       4) NO page numbers
%
% as against the acm_proc_article-sp.cls file which
% DOES NOT produce 1) thru' 3) above.
%
% Using 'sig-alternate.cls' you have control, however, from within
% the source .tex file, over both the CopyrightYear
% (defaulted to 200X) and the ACM Copyright Data
% (defaulted to X-XXXXX-XX-X/XX/XX).
% e.g.
% \CopyrightYear{2007} will cause 2007 to appear in the copyright line.
% \crdata{0-12345-67-8/90/12} will cause 0-12345-67-8/90/12 to appear in the copyright line.
%
% ---------------------------------------------------------------------------------------------------------------
% This .tex source is an example which *does* use
% the .bib file (from which the .bbl file % is produced).
% REMEMBER HOWEVER: After having produced the .bbl file,
% and prior to final submission, you *NEED* to 'insert'
% your .bbl file into your source .tex file so as to provide
% ONE 'self-contained' source file.
%
% ================= IF YOU HAVE QUESTIONS =======================
% Questions regarding the SIGS styles, SIGS policies and
% procedures, Conferences etc. should be sent to
% Adrienne Griscti (griscti@acm.org)
%
% Technical questions _only_ to
% Gerald Murray (murray@hq.acm.org)
% ===============================================================
%
% For tracking purposes - this is V2.0 - May 2012

\documentclass{sig-alternate}

\usepackage{comment}
\usepackage{cite}
\usepackage[shortlabels]{enumitem} 
\usepackage{amsmath}
\usepackage{url}
\usepackage{balance}
\newcommand{\bi}{\begin{itemize}[leftmargin=0.4cm]}
\newcommand{\ei}{\end{itemize}}
\newcommand{\be}{\begin{enumerate}}
\newcommand{\ee}{\end{enumerate}}
\newcommand{\tion}[1]{\S\ref{sect:#1}}
\newcommand{\fig}[1]{Figure~\ref{fig:#1}}
\setlist{nolistsep,leftmargin=5mm}
%\usepackage[pdftex]{graphicx}
\newcommand{\Sample}{{\bf SAMPLE}}
\newcommand{\Slope}{{\bf SLOPE}}
\usepackage{picture}
\usepackage{colortbl}
\usepackage[table]{xcolor}
\usepackage{listings}
%\usepackage[margin=1in]{geometry}

\definecolor{lightgray}{gray}{0.8}
\definecolor{darkgray}{gray}{0.6}


\definecolor{Gray}{gray}{0.95}
\definecolor{LightGray}{gray}{0.975}

\lstset{
    language=Python,
    basicstyle=\ttfamily\fontsize{2.4mm}{0.8em}\selectfont,
    breaklines=true,
    prebreak=\raisebox{0ex}[0ex][0ex]{\ensuremath{\hookleftarrow}},
    frame=tlrb,
    showtabs=false,
    showspaces=false,
    showstringspaces=false,
    %backgroundcolor=\color{Gray},
    keywordstyle=\bfseries,
    emph={COCONUT,GUESSES,ASSESS,COCOMO2,SLOPE,SAMPLE,WHERE,RIG}, emphstyle=\bfseries\color{Blue},
    stringstyle=\color{green!50!black},
    commentstyle=\color{red}\itshape,
    %numbers=none,
    captionpos=t,
    numberstyle=\bfseries\color{red},
    escapeinside={\%*}{*)}
}

\definecolor{darkgreen}{rgb}{0,0.3,0}

\usepackage[table]{xcolor}
\definecolor{Gray}{rgb}{0.88,1,1}
\definecolor{Gray}{gray}{0.85}
\definecolor{Blue}{RGB}{0,29,193}

\newcommand{\G}{\cellcolor{green}}
\newcommand{\Y}{\cellcolor{yellow}}


\newcommand{\quart}[4]{\begin{picture}(100,6)%1
{\color{black}\put(#3,3){\circle*{4}}\put(#1,3){\line(1,0){#2}}}\end{picture}}


\usepackage{times}

\def\baselinestretch{0.95}


\setlist{nosep}

 \usepackage[font={small}]{caption, subfig}



\setlength{\abovecaptionskip}{1ex}
 \setlength{\belowcaptionskip}{1ex}
 
 \setlength{\floatsep}{1ex}
 \setlength{\textfloatsep}{1ex}
 
\usepackage[compact,small]{titlesec}

\pagenumbering{arabic}
\begin{document}  
%
% --- Author Metadata here ---
\conferenceinfo{FSE}{'15 Bergamo, Italy}
%\CopyrightYear{2007} % Allows default copyright year (20XX) to be over-ridden - IF NEED BE.
%\crdata{0-12345-67-8/90/01}  % Allows default copyright data (0-89791-88-6/97/05) to be over-ridden - IF NEED BE.
% --- End of Author Metadata ---


\title{On the Value of Parametric Software Effort Estimation}


%
% You need the command \numberofauthors to handle the 'placement
% and alignment' of the authors beneath the title.
%
% For aesthetic reasons, we recommend 'three authors at a time'
% i.e. three 'name/affiliation blocks' be placed beneath the title.
%
% NOTE: You are NOT restricted in how many 'rows' of
% "name/affiliations" may appear. We just ask that you restrict
% the number of 'columns' to three.
%
% Because of the available 'opening page real-estate'
% we ask you to refrain from putting more than six authors
% (two rows with three columns) beneath the article title.
% More than six makes the first-page appear very cluttered indeed.
%
% Use the \alignauthor commands to handle the names
% and affiliations for an 'aesthetic maximum' of six authors.
% Add names, affiliations, addresses for
% the seventh etc. author(s) as the argument for the
% \additionalauthors command.
% These 'additional authors' will be output/set for you
% without further effort on your part as the last section in
% the body of your article BEFORE References or any Appendices.

\numberofauthors{5} %  in this sample file, there are a *total*
% of EIGHT authors. SIX appear on the 'first-page' (for formatting
% reasons) and the remaining two appear in the \additionalauthors section.
%
\author{
% You can go ahead and credit any number of authors here,
% e.g. one 'row of three' or two rows (consisting of one row of three
% and a second row of one, two or three).
%
% The command \alignauthor (no curly braces needed) should
% precede each author name, affiliation/snail-mail address and
% e-mail address. Additionally, tag each line of
% affiliation/address with \affaddr, and tag the
% e-mail address with \email.
%
% 1st. author
\alignauthor
Tim Menzies, \\
       \affaddr{CS, NcState, USA}\\
       {\scritpsize tim@menzies.us}
% 2nd. author
\alignauthor
Barry Boehm\\
       \affaddr{CS, USC, USA}\\
       {\scritpsize barryboehm@gmail.com}
% 3rd. author
\alignauthor Ye Yang\\
       \affaddr{Systems Eng., Stevens, USA}\\
       {\scritpsize yangye@gmail.com}
 % use '\and' if you need 'another row' of author names
% 4th. author
\and
\alignauthor Jairus Hihn\\
       \affaddr{JPL, Caltech, USA}\\
       {\scritpsize  jairus.hihn@jpl.nasa.gov}    
\alignauthor Naveen Lekkalapudi\\
       \affaddr{CS, WVU, USA}\\
       {\scritpsize nalekkalapudi@mix.wvu.edu}
\alignauthor George Mathew, \\
       \affaddr{CS, NcState, USA}\\
       {\scritpsize  george.meg91@gmail.com}
}
% There's nothing stopping you putting the seventh, eighth, etc.
% author on the opening page (as the 'third row') but we ask,
% for aesthetic reasons that you place these 'additional authors'
% in the \additional authors block, viz.
\additionalauthors{Additional authors: John Smith (The Th{\o}rv{\"a}ld Group,
email: {\texttt{jsmith@affiliation.org}}) and Julius P.~Kumquat
(The Kumquat Consortium, email: {\texttt{jpkumquat@consortium.net}}).}
\date{30 July 1999}
% Just remember to make sure that the TOTAL number of authors
% is the number that will appear on the first page PLUS the
% number that will appear in the \additionalauthors section.

\maketitle


\begin{figure}[!t]
{\scriptsize
{\bf NASA10 (new NASA data up to 2010):}


{\scriptsize \begin{tabular}{l@{~~}l@{~~}r@{~~}r@{~~}c}
\arrayrulecolor{darkgray}
\rowcolor[gray]{.9}  rank & treatment & median & IQR & %min= 0, max= 185\\
\\
  1 &      COCONUT &    34  &  14 & \quart{14}{8}{18}{111} \\
  1 &   COCOMO-II &    43  &  35 & \quart{13}{19}{23}{111} \\
\hline 
  2 &      TEAK &    73  &  80 & \quart{29}{43}{39}{111} \\
\hline  3 & Linear Regression &    83  &  142 & \quart{18}{77}{45}{111} \\
\end{tabular}}

~\\

{\bf COC05 (new COCOMO data up to 2005):}

{\scriptsize \begin{tabular}{l@{~~}l@{~~}r@{~~}r@{~~}c}
\arrayrulecolor{darkgray}
\rowcolor[gray]{.9}  rank & treatment & median & IQR & \\%min= 20, max= 300\\
  1 &      COCOMO-II &    46  &  134 & \quart{0}{41}{9}{110} \\
  1 & COCONUT &    62  &  38 & \quart{6}{6}{15}{110} \\
  2 &      TEAK &    84  &  110 & \quart{10}{32}{23}{110} \\
\hline 
  3 & Linear Regression &    117  &  253 & \quart{9}{83}{35}{110} \\
\end{tabular}}

% :learn 70.71 :analyze 4.04 :boots 2 effects 2 :conf 0.9801

~\\


{\bf NASA93 (NASA data up to 1993):}



{\scriptsize \begin{tabular}{l@{~~}l@{~~}r@{~~}r@{~~}c}
\arrayrulecolor{darkgray}
\rowcolor[gray]{.9}  rank & treatment & median & IQR & %min= 15, max= 250\\
\\
  1 &      COCONUT &    36  &  38 & \quart{0}{10}{9}{100} \\
  1 &      COCOMO-II &    39  &  39 & \quart{3}{10}{10}{100} \\
\hline  
  2 & TEAK &    50  &  81 & \quart{0}{28}{15}{100} \\
\hline
  3 & Linear Regression &    65  &  221 & \quart{3}{88}{21}{100} \\
\end{tabular}}

% :learn 548.2 :analyze 7.81 :boots 3 effects 14 :conf 0.970299



% :learn 37.86 :analyze 3.84 :boots 3 effects 5 :conf 0.970299
%\subsection{coc81}


~\\

{\bf COC81 (original data from the 1981 COCOMO book):}

{\scriptsize \begin{tabular}{l@{~~}l@{~~}r@{~~}r@{~~}c}
\arrayrulecolor{darkgray}
\rowcolor[gray]{.9}  rank & treatment & median & IQR & %min= 12, max= 500\\
\\
  1 &      COCOMO-II &    32  &  33 & \quart{2}{4}{4}{100} \\
  1 &      COCONUT &    33  &  42 & \quart{0}{6}{4}{100} \\
\hline  2 & TEAK &    93  &  128 & \quart{9}{24}{17}{100} \\
\hline  
  3 & Linear Regression &    351  &  1539 & \quart{14}{92}{40}{100} \\
\end{tabular}}

% :learn 260.57 :analyze 8.89 :boots 6 effects 11 :conf 0.941480149401

}
\caption{COCOMO vs COCONUT vs TEAK vs Linear Regression}\label{fig:fss}
\end{figure}

\end{document}
